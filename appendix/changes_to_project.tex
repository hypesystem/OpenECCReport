\section{Changes to Project Goals}

During the development of a project it is natural that some changes to the goals occur. The more research is done,
the more knowledge is acquired, and the better the project goals can be specified. For this project, the original
project statement was:

\begin{quote}
	The most efficient cryptographic protocols are symmetric, in that they use a single shared key. To securely exchange a shared key, public key cryptography is used. Public key cryptography relies on a public key for encryption and a private key for decryption, as first described by Diffie and Hellmann.

	Elliptic curves provide a new mathematical model that can be used as the basis for cryptography. Like its predecessors (relying on number theory), elliptic curves allow for a trapdoor one-way function to be constructed. Trapdoor functions are easy to compute but hard to revert.

	Through an adaption of Diffie and Hellmann’s original public-key protocol, it is possible to use elliptic curves to construct public key cryptography.

	Comparisons of the security provided per key-length for different cryptographic algorithms exist. With most mathematical operations - including those used in Elliptic Curve Cryptography (ECC) - several performance optimizations exist over the naïve implementation. The performance of traditional public-key cryptography implementations has been continuously improved since its inception. To understand whether ECC is feasible as an alternative to traditional cryptography, the performance of ECC must be evaluated in relation to that of traditional public-key cryptography implementations.
\end{quote}

While all the premises for the project still hold true, some ideas surrounding the comparisons changed:

\begin{itemize}
	\item The comparison of key-length per security would require an notion of the ``security'' for cryptographic protocols.
		This was deemed out of scope.
	\item The performance of non-EC cryptosystems was not measured, as ECC implementations already exist, and are used. It was
		more relevant to compare OpenECC to these implementations (more specifically, Bouncy Castle).
	\item A few additional comparisons were added: relative performance of multiplication algorithms; and an analysis of which
		parts of the system uses the most processing time.
\end{itemize}