\section{Future Work on OpenECC}

OpenECC is currently at a stage where it can be considered mostly a proof-of-concept, showing the implementation
of several ECC algorithms and constructs without considering \emph{real} security (against, eg. timing attacks).

There are two angles from which the project can be improved. The most obvious is the number of features offered by
the library. Currently, several layers of the project only support a single algorithm -- but several different
possibilities exist.

The second angle is more theoretical, focusing on establishing a more rigorous practice of security vetting of the
library. Requirements must be set for the quality and safety of the implementation. This is an extensive undertaking
that is out of scope of any in-depth discussion in this project.

\subsection{Features}

Following is a list of some features that have been considered for implementation or discussed peripherally during the
project. Many more possible features exist.

\begin{description}
    \item[Finite Field Implementation] Currently OpenECC depends on Bouncy Castle's implementation of finite fields.
        Removing this dependency should be a priority, as it would make the size of the library required to do ECC
        much smaller.
    \item[Binary representation of Points] While the current implementation of OpenECC works well within its own world,
        it is not usable in a real-world context, as the points cannot be represented in a form such that they are
        transmissible. Implementing this would make points, ciphertexts and plaintexts transferable as bit strings.
    \item[Montgomery Curves] Montgomery curves have several advantages, most importantly the ladder allowing for constant-time
        scalar multiplication. Edwards curves may also be considered.
    \item[Deterministic Encoding] The current encoding algorithms has severe limits on message length. Solving this problem,
        and at the same time making sure that any message can always be mapped to a point on the curve is important for
        OpenECC to be usable in the real world (where longer messages or keys are transmitted).
    \item[Signing algorithms] Signing of messages is used many places to verify authenticity. Algorithms such as the Digital
        Siginature Algorithm (DSA) have been adapted to elliptic curves (ECDSA), and could be implemented to offer this
        functionality.
    \item[Hybrid encryption] For encryption of longer messages, asymmetric encryption is quicker. Hybrid encryption takes
        advantage of this. Implementing ECIES would offer some hybrid encryption functionality.
\end{description}