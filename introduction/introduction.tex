\section{Introduction}

The most efficient cryptographic protocols are symmetric, meaning that they use a single shared key.
To securely exchange a shared key, public key cryptography is used. Public key cryptography
relies on a public key for encryption and a private key for decryption.

Elliptic Curve Cryptography (ECC) is based on the properties of elliptic curves.
An elliptic curve is a curve \(E\) that is symmetric around the x-axis (see Figure \ref{fig:graphs}),
and can be described with the formula:

\begin{equation} \label{eq:elliptic-curve-formula-full}
	E: y^2 + a_1xy + a_3y = x^3 + a_2x^2 + a_4x + a_6
\end{equation}

Any elliptic curve can have its formula reduced to a much simpler form, which -- unlike the original
form -- is used in cryptography (see Section \ref{sec:math_curve}):

\begin{equation}
	E: y^2 = x^3 + ax + b
\end{equation}

When using elliptic curves for cryptography, a large prime \(p\) is selected, and the curves are defined
over a prime field \(\mathbb{F}_p\) of integers.

\begin{figure}[htb]
	\centering
	\includegraphics[width=0.37\textwidth]{introduction/secp256k1-graph}
	\includegraphics[width=0.52\textwidth]{introduction/secp256k1-graph-over-field-p17}
	\caption{The curve \(E: y^2 = x^3 + 0x + 7\) depicted left, and the same curve over the prime field
		\(\mathbb{F}_{17}\) depicted right.}
	\label{fig:graphs}
\end{figure}

Each curve, in addition to all the regular points, contains a point at infinity, \(O_\infty\). For example, a
curve \(E: y^2 = x^3 + 7\) over a prime field \(\mathbb{F}_{17}\) can be written as:

\begin{equation}
	E(\mathbb{F}_{17}) = \{ O_\infty; (1,5); (1,12); ...; (15,13) \}
\end{equation}

ECC relies on the hardness of the Elliptic Curve Discrete Logarithm Problem (ECDLP) to be secure. ECDLP is
defined as follows: ``given an elliptic curve \(E\) defined over a finite field \(\mathbb{F}_q\), a point
\(P \in E(\mathbb{F}_q)\) of order \(n\), and a point \(Q \in \langle P \rangle\), find the integer
\(l \in [0,n-1]\) such that \(Q = lP\). The integer \(l\) is called the \emph{discrete logarithm of
\(Q\) to the base \(P\)}, denoted \(l = log_P Q\).''\footnote{The hardness of the adapted ElGamal (see
Section \ref{sec:math_encryption}) is exactly that of the ECDLP, with the public key \(Q\) and private
key \(l\).}\cite{hankerson2010}

The order \(n\) of \(P\) is the smallest possible number such that \(nP = O_{\infty}\) (the point at infinity), and is used to describe
the cyclic subgroup \(\langle P \rangle = \{ O_\infty, P, 2P, ..., (n-1)P \} \). For a generator, this cyclic subgroup
will contain all of the points on the curve.

There are several other concerns than the ECDLP to be considered when evaluating the safety of using a curve. Bernstein and Lange
provide an extensive list of requirements for the safety of curves, including indistinguishability from uniform random
strings, and ``simple, fast, constant-time, single-coordinate single-scalar multiplication''. They also list a variety
of curves, noting which of the requirements they fulfil.\cite{safecurves}

\subsection{In this report}

Elliptic curves provide a new mathematical model that can be used as the basis for cryptography. Several
implementations of ECC already exist: Bouncy Castle is a free Java and C\# implementation of various cryptosystems,
including some based on elliptic curves.\cite{bouncycastle}

This report explores the mathematical foundations and algorithms of ECC, from curves to encryption; and
describes the structure of \emph{OpenECC}, an open-source implementation of the mathematics and algorithms.
OpenECC has been developed concurrently with the production of the report, and is used as a basis for performance
measurements.

Arithmetic using elliptic curves of the simplest (Weierstrass) form is explained (Section \ref{sec:math_curve}) and
several algorithms for scalar point multiplication are discussed (including NAF and wNAF, Section
\ref{sec:math_curve_multiplication}).

In 1995, ElGamal described a simple (as in simply brilliant) public key cryptosystem that can be adapted
to use elliptic curves (Section \ref{sec:math_encryption_elgamal}). This system relies on the ability to
encode messages as points on an elliptic curve (Section \ref{sec:math_encoding}).

The implementation is modular, allowing for different curves and encryption schemes to be swapped in and out
rapidly (Section \ref{sec:implementation}). As such, the library can be extended in the future to support different
curves, encodings, and encryption schemes. It also provides a readable syntax that is closer to natural mathematical
syntax than available alternatives, such as Bouncy Castle (Section \ref{sec:implementation_curves}).

While OpenECC implements the constructs used specifically in elliptic curve cryptography, the elliptic curves rely on
Bouncy Castle library's implementation of finite fields (Section \ref{sec:implementation_dependency}).

Several multiplication algorithms are compared with each other (Section \ref{sec:performance_multipliers});
the encryption of a string is broken down and the distribution of time spent in different parts of the program
is discussed (Section \ref{sec:performance_components}); and finally, OpenECC is compared to Bouncy Castle to
get an indication of the real-world applicability of the implementation (Section \ref{sec:performance_bouncycastle}).

OpenECC is by no means a complete library that should be used in production. It should not be used to encrypt real,
sensitive data (see Appendix \ref{app:disclaimer}).

The code for OpenECC can be found at \texttt{https://github.com/hypesystem/OpenECC}, and it is recommended to look
through it while to see concrete implementations of the algorithms and calculations described in Section
\ref{sec:math} and to see how the architecture described in Section \ref{sec:implementation} dictates the design
of the library.