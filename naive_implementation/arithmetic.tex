\subsubsection{Arithmetic}

Note: Illustrations of the idea here would be nice.

The addition, \(P + Q\), is done by drawing a line through the points, \(P\) and \(Q\),
and finding the point \(R_{reflection})\), where the line intersects the curve again. This point is
then reflected over the x-axis to find \(R\), the result. (Guide to Elliptic Curve Crypto... p. 79)

The double of \(P\), \(2P\) or \(P+P\), is found by drawing the tangent of the curve at \(P\).
\(R_{reflection}\) is once again found at the line's (in this case the tangent's) next intersection
with the curve, and the result \(R\), is found by reflecting \(R_{reflection}\). (Guide to Elliptic Curve Crypto... p.79)

Group law for \(E/K\), integers in K, where K is a (finite) field.
\(y^2 = x^3 + ax + b\), char(K) not= 2,3

\(P + \infty = P\) (Adding a point to infinity yields the point)

\(P=(x,y) \implies -P=(x,-y)\) (negation is done by negating y)

\(P - P = \infty\) (a point subtracted from itself yields infinity)

\(-\infty = \infty\) (negative infinity equals infinity)

\paragraph{Addition}

Actual addition (Guide to elliptic curve crypto... p. 80)
\begin{quote}
	3. Point addition. Let P = (x1,y1) in E(K) and Q = (x2,y2) in E(K), where P not= -+ Q.
	Then P+Q = (x3,y3), where
	
	x3 = ((y2-y1)/(x2-x1))\^2 - x1 - x    and    y3 = ((y2-y1)/(x2-x1))*(x1-x3)-y1
\end{quote}

Implementation
\begin{verbatim}
	public override Point Add(Point q)
	{
	    //p + infinity = p
	    if (q == Infinity)
	        return this;

	    //p + -p = infinity
	    if (this == -q)
	        return Infinity;
\end{verbatim}

\paragraph{Doubling}

Used if \(Q = P\) in \(P + Q\). Guide... p. 80 (quote)
\begin{quote}
	Let P = (x1,y1) in E(K), where P not= -P. Then 2P = (x3,y3), where
	
	x3 = ((3x1\^2+a)/(2y1))\^2-2x1    and    y3 = ((3x1\^2+a)/2y1)*(x1-x3)-y1
\end{quote}