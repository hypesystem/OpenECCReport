\subsection{Arithmetic}

Note: Illustrations of the idea here would be nice.

Two points on the curve, P and Q, are added by drawing a line through P and Q. A point R\_reflection
is found, where the line intersects the curve again. The reflection of R\_reflection, R, is found as
the result of the addition. (Guide to Elliptic Curve Crypto... p. 79)

The double of P, 2P or P+P, is found by drawing the tangent of the curve at P. R\_reflection is once again found at the line's (in this case the tangent's) next intersection with the curve, and the result R, is found by reflecting R\_reflection. (Guide to Elliptic Curve Crypto... p.79)

Group law for E/K (integers in K, where K is most likely a finite field)
y\^2 = x\^3 + ax + b, char(K) not= 2,3

Adding a point to infinity yields the point.

(x,y) + (x,-y) = infinity. If P=(x,y) then -P=(x,-y) (negative P).

-infinity = infinity (negative infinity equals infinity).

\subsubsection{Addition}

Actual addition (Guide to elliptic curve crypto... p. 80)
\begin{quote}
	3. Point addition. Let P = (x1,y1) in E(K) and Q = (x2,y2) in E(K), where P not= -+ Q.
	Then P+Q = (x3,y3), where
	
	x3 = ((y2-y1)/(x2-x1))\^2 - x1 - x    and    y3 = ((y2-y1)/(x2-x1))*(x1-x3)-y1
\end{quote}

Implementation
\begin{verbatim}
    public override Point Add(Point q)
    {
        //p + infinity = p
        if (q == Infinity)
            return this;

        //p + -p = infinity
        if (this == -q)
            return Infinity;
\end{verbatim}

\subsubsection{Doubling}

Guide... p. 80 (quote)
\begin{quote}
	Let P = (x1,y1) in E(K), where P not= -P. Then 2P = (x3,y3), where
	
	x3 = ((3x1\^2+a)/(2y1))\^2-2x1    and    y3 = ((3x1\^2+a)/2y1)*(x1-x3)-y1
\end{quote}