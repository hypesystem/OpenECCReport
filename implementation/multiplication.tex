\subsection{Multiplication}

Note: get sources! (Hankerson?)

Note: Consider moving the Multiplication section to an appendix --- is it relevant enough to go in the report?

The multiplication \(dP\) where \(d\) is an integer and \(P\) is a point occurs very often in ECC. Multiplication
is easy to do in a naive and very slow way. The simplest is simply adding the point to itself until the result is
reached:

\begin{verbatim}
public override Point Multiply(BigInteger x)
{
  Point result = this;
  for (int i = 1; i < x; i++)
  {
    result = result + this;
  }
  return result;
}
\end{verbatim}

This, however, is horribly slow. Fortunately there are several ways to speed it up.

Note: Point multiplication is covered in Hankerson pp. 95-?

\paragraph{Double-and-Add}

Double-and-Add takes advantage of the fact that doubling an integer is just adding it to itself. Addition is easy,
and taking advantage of this, multiplication can be achieved with much fewer additions. The idea is to double the
result if the remaining number of multiplications is even, and simply adding the original value if it is not.

\begin{verbatim}
public override Point Multiply(BigInteger x)
{
  Point result = this;
  while(x > 0)
  {
    if(x % 2 == BigInteger.Zero)
    {
      result = result + result;
      x = x/2;
    }
    else {
      result = result + this;
      x--;
    }
  }
  return result;
}
\end{verbatim}

This improves performance somewhat. However, more can be done.

\paragraph{Windowed method}

The Windowed method ... \verb+http://en.wikipedia.org/wiki/Elliptic_curve_point_multiplication#Windowed_method+