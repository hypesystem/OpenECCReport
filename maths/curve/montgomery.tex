\subsubsection{Montgomery Curves}

Note: Explain in more detail the safety of curves.

Most safe curves\footnote{According to http://safecurves.cr.yp.to/} are not of the supported simple
Weierstrass form, but are rather represented as \emph{Edwards} or \emph{Montgomery} curves. Both of
these types of curves are more recent than the simple Weierstrass forms, and have shown to have some
desireable properties\footnote{Source? Expand on this.}

Note: Why have I chosen Montgomery and not Edwards curves?

Montgomery curves have the following form:

\begin{equation}
	y^2 = x^3 + ax^2 + x
\end{equation}

It is possible to represent all curves in Weierstrass simple forms, and as such it is also possible to
transform any of the safe Montgomery curves to this form.\cite{safecurves}
This does not, however, come without a cost, as the Weierstrass form of a Montgomery curve will have
very large integers as its parameters, compared to those in the original Montgomery form. For example,
the Montgomery curve, \verb|Curve25519| has the equation \(y^2 = x^3+486662x^2+x\), but
translated into the Weierstrass simple form, it will have an equation that does not fit on the width of
a normal A4 page:\footnote{See the full calculations and equation in Appendix \ref{app:montgomery_weierstrass}.
{\bf Note: Appendix with calculations, due to Lorena!}}

\begin{equation}
	y^2 =
	x^3 +
	19298681539552699237261830834781317975544997444273427339909597334573241639236x +
	55751746669818908907645289078257140818241103727901012315294400837956729358436
\end{equation}

Note: What is "acceptable change in parameters"? (The "rule" that is used to transform curves to Weierstrass form.) (Guide to Elliptic Curve Cryptography)

Note: In addition to smaller parameters, Montgomery curves have other inherent factors that result in quicker
computations. List them here.

Note: Operations on this curve are less trivial. It is also not supported by BouncyCastle.

Note: Arithmetic section here :)

\paragraph{Addition}

\paragraph{Doubling}

\paragraph{Subtraction}