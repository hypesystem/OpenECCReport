\subsubsection{Other Curve Forms}

There are more efficient representations of elliptic curves than the simple
Weierstrass form. Both Edwards or Montgomery curves are more recent than the simple Weierstrass forms,
and have shown to have some desirable properties.\cite{safecurves}

Neither Montgomery nor Edwards curves are supported in OpenECC, but they can be converted to the
usable Weierstrass form. Montgomery curves have the following formula:

\begin{equation}
	y^2 = x^3 + ax^2 + x
\end{equation}

Whereas it is possible to represent all curves in Weierstrass simple forms, the same cannot be said for
the Montgomery form (or the Edwards form for that matter). It is, however, possible to
transform any of the safe Montgomery curves to the Weierstrass form.\cite{safecurves}

This does not come without a cost as the Weierstrass form of a Montgomery curve will have
very large integers as its parameters, compared to those in the original Montgomery form. For example,
the Montgomery curve, \verb|Curve25519| has the equation \(y^2 = x^3+486662x^2+x\), but
translated into the Weierstrass simple form, it will have an equation that does not fit on the width of
a normal A4 page:\footnote{See the full calculations and equation in Appendix \ref{app:montgomery_weierstrass}.}

\begin{equation}
	y^2 =
	x^3 +
	19298681539552699237261830834781317975544997444273427339909597334573241639236x +
	55751746669818908907645289078257140818241103727901012315294400837956729358436
\end{equation}

These large parameters make Montgomery curves infeasible to use, as multiplication would be very slow. As such,
it is discouraged to use curves that are not normally in the Weierstrass form.\footnote{Unless support for Montgomery
curves is implemented in the future, which is enabled by the extendable structure of OpenECC (see Section
\ref{sec:implementation_curves}).}