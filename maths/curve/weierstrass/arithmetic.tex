\paragraph{Addition}

Addition of two points \(P + Q = R\) is a fairly straight-forward matter. Given two points, \(P = (x_1,y_1)\) and
\(Q = (x_2,y_2)\), the result \(R = (x_3,y_3)\) can be found with the following calculations:

\begin{equation}
	\lambda = {{y_2-y_1} \over {x_2-x_1}}
\end{equation}

\begin{equation}
	x_3 = \lambda^2 - x_1 - x_2 \textbf{ and } y_3 = \lambda (x_1 - x_3) - y_1
\end{equation}

As elliptic curves also contain the \(\infty\) (infinity) point, there is a single exception to this
rule: if either \(P\) or \(Q\) is infinity, the result of adding them will be infinity, too.

\begin{equation}
    Q = \infty \lor P = \infty \implies P + Q = \infty\
\end{equation}

\paragraph{Doubling}

If the points being added are equal to one another (\(P + Q = R \text{, where } P = Q \text{ and } P = (x_1,y_1)\)),
the result can be calculated with a simplified formula:

\begin{equation}
	\lambda = {{3x_1^2 + a} \over {2y_1}} \text{, where } a \text { is the parameter from the curve formula }
\end{equation}

\begin{equation}
	x_3 = \lambda^2 - 2x_1 \textbf{ and } y_3 = \lambda (x_1 - x_3) - y_1
\end{equation}

The resulting point is then \(R = (x_3, y_3)\).

\paragraph{Subtraction}

A subtraction \(P - Q = R\) can be expressed through addition and negation, as \(P - Q = P + (-Q)\). Negation of a given point
\(P = (x,y)\) is calculated as \(-P = (x,-y)\) (a reflection over the x-axis).

As with addition, negation is complicated by the existence of the infinity point:

\begin{equation}
	-\infty = \infty \textbf{ and } P - P = \infty
\end{equation}

The negation of infinity is infinity (negative infinity does not exist), and a point subtracted from itself is infinity.\cite{hankerson2010}

\paragraph{Scalar Multiplication}

Scalar multiplication of a point \(P\) and an integer \(d\) is calculated through repeated addition of the point to an accumulator.
Several algorithms to speed up this process exist. Multiplication is the slowest operation performed in ECC, and its performance should
be considered carefully.\footnote{The DoubleAndAdd method limits the number of additions necessary by doubling the accumulating result
whenever possible, instead of naively adding \(P\); \textbf{Note: Explain FpNaf...}. Both of these methods are supported in OpenECC, and
the implementation can easily be extended with more multiplication algorithms (see Section \ref{sec:implementation_multipliers}).}