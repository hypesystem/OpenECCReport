\paragraph{Addition}

Addition of two points \(P + Q = R\) is a fairly straight-forward matter. Given two points, \(P = (x_1,y_1)\) and
\(Q = (x_2,y_2)\), the result \(R = (x_3,y_3)\) can be found with the following calculations:

\begin{equation}
	\lambda = {{y_2-y_1} \over {x_2-x_1}}
\end{equation}

\begin{equation}
	x_3 = \lambda^2 - x_1 - x_2 \textbf{ and } y_3 = \lambda (x_1 - x_3) - y_1
\end{equation}

As elliptic curves also contain the \(\infty\) (infinity) point, there is a single exception to this
rule: \(Q = \infty \lor P = \infty \implies P + Q = \infty\) (if either \(P\) or \(Q\) is infinity,
the result of adding them will be infinity, too).

\paragraph{Doubling}

If the points being added are equal to one another (\(P + Q = R \text{, where } P = Q\)), the result on the curve of form
\(y^2 = x^3 + ax + b\) can be calculated with a simplified formula:

\begin{equation}
	\lambda = {{3x_1^2 + a} \over {2y_1}} \text{, where } a \text { is the parameter from the curve formula }
\end{equation}

\begin{equation}
	x_3 = \lambda^2 - 2x_1 \textbf{ and } y_3 = \lambda (x_1 - x_3) - y_1
\end{equation}

\paragraph{Subtraction}

A subtraction \(P - Q = R\) can be expressed through addition and negation, as \(P - Q = P + (-Q)\). Negation of a given
\(P = (x,y)\) is calculated as \(-P = (x,-y)\). Subtraction can then be calculated as \((x_1,y_1) - (x_2,y_2) = (x_3,-y_3)\).

As with addition, negation is complicated by the existence of the infinity point. The negation of infinity is infinity (negative
infinity does not exist), and a point subtracted from itself is infinity:

\begin{equation}
	-\infty = \infty \textbf{ and } P - P = \infty
\end{equation}

Note: All of the above is from Hakerson p. 80.