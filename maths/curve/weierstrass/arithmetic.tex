\paragraph{Addition}

Addition of two points \(P + Q = R\) is a fairly straight-forward matter. Given two points, \(P = (x_1,y_1)\) and
\(Q = (x_2,y_2)\), the result \(R = (x_3,y_3)\) can be found with the following calculations:

\begin{align}
	& x_3 = \lambda^2 - x_1 - x_2 \textbf{ and } y_3 = \lambda (x_1 - x_3) - y_1\\
    & \textbf{where } \lambda = {{y_2-y_1} \over {x_2-x_1}}
\end{align}

As elliptic curves also contain the point at infinity \(O_\infty\) the exceptions explained above apply:

\begin{equation}
	P + O_\infty = P
\end{equation}

\begin{equation}
	P + (-P) = O_\infty
\end{equation}

\paragraph{Doubling}

If the points being added are equal to one another (\(P + Q = R \text{, where } P = Q \text{ and } P = (x_1,y_1)\)),
the result can be calculated with a simplified formula:

\begin{align}
	& x_3 = \lambda^2 - 2x_1 \textbf{ and } y_3 = \lambda (x_1 - x_3) - y_1 \\
    & \textbf{where } \lambda = {{3x_1^2 + a} \over {2y_1}} \text{, with } a \text { being the parameter from the curve formula }
\end{align}

The resulting point is then \(R = (x_3, y_3)\).

\paragraph{Subtraction}

A subtraction \(P - Q = R\) can be expressed through addition and negation, as \(P - Q = P + (-Q)\). Negation of a given point
\(P = (x,y)\) is calculated as \(-P = (x,-y)\) (a reflection over the x-axis).

As with addition, negation is complicated by the existence of the point at infinity:

\begin{equation}
	-O_\infty = O_\infty \textbf{ and } P - P = O_\infty
\end{equation}

The negation of the point at infinity is the point at infinity (negative infinity does not exist), and a point subtracted from
itself is the point at infinity.\cite{hankerson2010}

\subsubsection{Scalar Multiplication}
\label{sec:math_curve_multiplication}

Scalar multiplication of a point \(P\) and an integer \(d\) is, with the most simple algorithm, calculated through repeated addition
of the point to an accumulator. Several algorithms to speed up this process exist, and as point multiplication is where most computational
time is spent\footnote{See Section \ref{sec:performance_components} for a confirmation of this.} a few are covered in the following:
\emph{Double-and-Add}, \emph{NAF}, and \emph{wNAF}.

While these algorithms give some idea of the advancement that can be made in both complexity and theoretical running time, they are not
the most advanced or well-performing algorithms known.\cite{hankerson2010}

\paragraph{Double-and-Add}

The DoubleAndAdd method limits the number of additions necessary by doubling the accumulating result whenever possible, resulting in
fewer total additions.

The algorithm relies on first representing the number \(d\) in its binary form, \(d_{binary} = (d_{t-1}, ... , d_2, d_1, d_0)_2\),
where \(t\) is the number of bits used to represent the number. Each bit is then inspected, from least significant (right) to most
significant (left): if the bit is \(1\), the current value of \(P\) is added to the accumulator. For each bit inspected, \(P\) is
doubled.\cite{hankerson2010}

This algorithm results in \(~1.5t\) additions (most of which are doublings), much less than the \(~0.5t^2\) additions required for
repeatedly adding the point to an accumulator.

\paragraph{Non-adjacent form}

Whereas the binary form of a number is constructed from \(\{0,1\}\), the non-adjacent form (NAF) is constructed from \(\{-1,0,1\}\), with
the added limitation that no non-zero values are adjacent (hence the name). Subtraction of points is (almost) as efficient as addition,
requiring only an additional negation of a finite field element (see \ref{sec:math_curve} \emph{Subtraction}).

The NAF of a number \(d\) can be constructed by an algorithm, which finds each digit as \(d_i = 2 - (d \text{ mod } 4)\) if \(d\) is odd, and
\(0\) otherwise (see Figure \ref{fig:compute-naf-algorithm}).

\begin{figure}[htb!]
	\centering
	\begin{tabular}{|p{\textwidth}|}
		\hline
		Input: Positive integer \(k\). \\
		Output: \(\text{NAF}_k\).

		\begin{enumerate*}
			\item \(i \gets 0\).
            \item While \(k \geq 1\) do
			\begin{enumerate*}
                \item If \(k\) is odd then: \(k_i \gets 2 - (k \text{ mod } 4)\), \(k \gets k - k_i\);
                \item Else: \(k_i \gets 0\).
                \item \(k \gets k/2\), \(i \gets i + 1\).
			\end{enumerate*}
			\item Return(\(k_{i-1}, k_{i-2}, ..., k_1, k_0\)).
		\end{enumerate*}
		\\
		\hline
	\end{tabular}
	\caption{The algorithm for computing the non-adjacent form of an integer \(k\), as represented by Hankerson et.al.\cite{hankerson2010}}
	\label{fig:compute-naf-algorithm}
\end{figure}

The multiplication \(dP\) can be calculated using the \(\text{NAF}_d\) with an appropriate algorithm (see Figure \ref{fig:naf-algorithm}).

\begin{figure}[htb!]
	\centering
	\begin{tabular}{|p{\textwidth}|}
		\hline
		Input: Positive integer \(k\), \(P \in E(\mathbb{F}_q)\). \\
		Output: \(kP\).

		\begin{enumerate*}
			\item Compute \(\text{NAF}_k = \Sigma^{l-1}_{i=0} k_i2^i\).
			\item \(Q \gets \infty\).
			\item For \(i\) from \(l-1\) downto \(0\) do
			\begin{enumerate*}
				\item \(Q \gets 2Q\).
				\item If \(k_i = 1\) then \(Q \gets Q + P\).
				\item If \(k_1 = -1\) then \(Q \gets Q - P\).
			\end{enumerate*}
			\item Return(\(Q\)).
		\end{enumerate*}
		\\
		\hline
	\end{tabular}
	\caption{The algorithm for computing the scalar multiplication \(kP\) using the non-adjacent form of \(k\), as represented by
		Hankerson et.al.\cite{hankerson2010}}
	\label{fig:naf-algorithm}
\end{figure}

The running time for the NAF algorithm is approximately \({4 \over 3} l\), where \(l\) is the length of the NAF, quicker than that
of the Double-and-Add.\cite{hankerson2010}

\paragraph{Windowed NAF}

The NAF can be improved further by allowing a wider range of values than just those in \(\{-1,0,1\}\). The range is called a \emph{window}, and
using this windowed non-adjacent form (wNAF) can improve running time, but requires some pre-computation. Regular NAF is the same
as \(\text{wNAF}_2\) (a window size of 2), and \(\text{wNAF}_3\) would be constructed from the values \(\{-3,-1,0,1,3\}\).

The wNAF of an integer is calculated in much the same way as a NAF, but with some added complexity due to the variable size of \(w\)
(see Figure \ref{fig:compute-wnaf-algorithm}).

\begin{figure}[htb!]
	\begin{tabular}{|p{\textwidth}|}
		\hline
		Input: Window width \(w\), positive integer \(k\). \\
		Output: \(\text{wNAF}_w(k)\).
		\begin{enumerate*}
			\item \(i \gets 0\).
			\item While \(k \geq 1\) do
			\begin{enumerate*}
				\item If \(k\) is odd then: \(k_i \gets k \text{ mods } 2^w\), \(k \gets k - k_i\);
				\item Else: \(k_i \gets 0\).
				\item \(k \gets k/2\), \(i \gets i + 1\).
			\end{enumerate*}
			\item Return(\(k_{i-1},k_{i-2},...,k_1,k_0\)).
		\end{enumerate*} \\
		\hline
		Where \(k \text{ mods } 2^w\) is: 
		\begin{enumerate*}
			\item If \(k \text{ mod } 2^w \geq 2^{w-1}\) then return(\((k \text{ mod } 2^w) - 2^w\));
			\item Else return(\(k \text{ mod } 2^w\)).
		\end{enumerate*} \\
		\hline
	\end{tabular}
	\caption{The algorithm for constructing a wNAF representation of an integer \(k\), as represented by Hankerson
		et.al.\cite{hankerson2010} Here with added explanation of \(k \text{ mods } 2^w\).}
	\label{fig:compute-wnaf-algorithm}
\end{figure}

The wNAF multiplication algorithm relies on the precomputation of the values \(P_i = iP \text{ for } i \in \{1,3,5,...,2^{w-1}-1\}\). If the
point \(P\) is fixed for many operations, then this precomputation may be worthwhile, even at big values for \(w\). For dynamically varying
\(P\) (such as in asymmetric encryption) a large \(w\) may result in an overhead larger than the benefit of the quicker operations.

The wNAF multiplication algorithm is very similar to that of the regular NAF, distinguishing itself by its precomputation and slightly more
advanced structure for non-zero \(k_i\) (see Figure \ref{fig:wnaf-algorithm}).

\begin{figure}[htb!]
	\begin{tabular}{|p{\textwidth}|}
		\hline
		Input: Window width \(w\), positive integer \(k\), \(P \in E(\mathbb{F}_q)\).\\
		Output: \(kP\).
		\begin{enumerate*}
			\item Compute \(\text{wNAF}_w(k) = \Sigma^{l-1}_{i=0} k_i 2^i\).
			\item Compute \(P_i = iP\) for \(i \in \{1,3,5,...,2^{w-1}-1\}\).
			\item \(Q \gets \infty\).
			\item For \(i\) from \(l - 1\) downto \(0\) do
			\begin{enumerate*}
				\item \(Q \gets 2Q\).
				\item If \(k_i \neq 0\) then:
				\begin{enumerate*}
					\item If \(k_i > 0\) then \(Q \gets Q + P_{k_i}\);
					\item Else \(Q \gets Q - P_{-k_i}\).
				\end{enumerate*}
			\end{enumerate*}
			\item Return(\(Q\)).
		\end{enumerate*} \\
		\hline
	\end{tabular}
	\caption{Algorithm for computing the multiplication \(kP\) using the wNAF algorithm, as represented by Hankerson et.al.\cite{hankerson2010}}
	\label{fig:wnaf-algorithm}
\end{figure}

wNAF is a generalization of NAF, and so the formula for running time of wNAF is a generalization of the running time for NAF. The number of
additions performed in wNAF depends on the window \(w\) and can be described as \(2^{w-2} + (1 + {1 \over {w + 1}}) l\), where \(l\) is the length
of the wNAF form.\cite{hankerson2010}