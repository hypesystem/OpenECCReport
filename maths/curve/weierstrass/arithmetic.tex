\paragraph{Addition}

Addition of two points \(P + Q = R\) is a fairly straight-forward matter. Given two points, \(P = (x_1,y_1)\) and
\(Q = (x_2,y_2)\), the result \(R = (x_3,y_3)\) can be found with the following calculations:

\begin{equation}
	\lambda = {{y_2-y_1} \over {x_2-x_1}}
\end{equation}

\begin{equation}
	x_3 = \lambda^2 - x_1 - x_2 \textbf{ and } y_3 = \lambda (x_1 - x_3) - y_1
\end{equation}

As elliptic curves also contain the \(\infty\) (infinity) point, there is a single exception to this
rule: if either \(P\) or \(Q\) is infinity, the result of adding them will be infinity, too.

\begin{equation}
    Q = \infty \lor P = \infty \implies P + Q = \infty\
\end{equation}

\paragraph{Doubling}

If the points being added are equal to one another (\(P + Q = R \text{, where } P = Q \text{ and } P = (x_1,y_1)\)),
the result can be calculated with a simplified formula:

\begin{equation}
	\lambda = {{3x_1^2 + a} \over {2y_1}} \text{, where } a \text { is the parameter from the curve formula }
\end{equation}

\begin{equation}
	x_3 = \lambda^2 - 2x_1 \textbf{ and } y_3 = \lambda (x_1 - x_3) - y_1
\end{equation}

The resulting point is then \(R = (x_3, y_3)\).

\paragraph{Subtraction}

A subtraction \(P - Q = R\) can be expressed through addition and negation, as \(P - Q = P + (-Q)\). Negation of a given point
\(P = (x,y)\) is calculated as \(-P = (x,-y)\) (a reflection over the x-axis).

As with addition, negation is complicated by the existence of the infinity point:

\begin{equation}
	-\infty = \infty \textbf{ and } P - P = \infty
\end{equation}

The negation of infinity is infinity (negative infinity does not exist), and a point subtracted from itself is infinity.\cite{hankerson2010}

\subsubsection{Scalar Multiplication}

Scalar multiplication of a point \(P\) and an integer \(d\) is, with the most simple algorithm, calculated through repeated addition
of the point to an accumulator. Several algorithms to speed up this process exist, and as point multiplication is most computational
time is spent\footnote{\textbf{Note: cite?}} a few are covered in the following: \emph{Double-and-Add}, \emph{NAF}, and \emph{wNAF}.

\paragraph{Double-and-Add}

The DoubleAndAdd method limits the number of additions necessary by doubling the accumulating result whenever possible, resulting in
fewer total additions. The algorithm relies on first representing the number \(d\) in its binary form, \textbf{Note: bin form here}.
Each bit is then inspected, from least significant (right) to most significant (left): if the bit is \(1\), the current value of \(P\)
is added to the accumulator. For each bit inspected, \(P\) is doubled.

\textbf{Note: Algorithm}

\textbf{This results in... (\# additions), fewer than naive \# additions}

\paragraph{Non-adjacent form, NAF}

\textbf{Note: Write this}

\paragraph{windowed-NAF}

\textbf{Note: introduce wNAF\_3 as syntax for wNAF with width 3.}