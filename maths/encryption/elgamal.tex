\label{sec:math_encryption_elgamal}

In 1985, ElGamal was proposed as a public key system, including a key distribution scheme based
on the difficulty of computing discrete logarithms. It allows messages to be sent from a sender
\(A\) to a recipient \(B\) to be encrypted in such a way that only \(B\) can decrypt them.

The sender and recipient agree on a common finite field (over a large prime \(p\)) to work in,
and use the same generator \(g\). \(B\) randomly selects a secret key \(d_B\) and generates a
public key \(q_B = g^{d_B}\). The public key is distributed to the sender. Given a message \(m\),
\(A\) can now construct a ciphertext \(c\) from a randomly selected \(k\):\cite{elgamal}

\begin{equation}
	c = (c_1, c_2)  \textbf{ where }  c_1 = g^k \land c_2 = m q_B^k
\end{equation}

\(B\) can find \(q_B^k\) by using \(c_1\) and his own secret key:

\begin{equation}
	q_B^k = g^{d_B k} = c_1^{d_B}
\end{equation}

Finding the original message \(m\) is then as simple as:

\begin{equation}
    m = {c_2 \over c_1^{d_B}}
\end{equation}

\paragraph{Adapting to Elliptic Curves}
Hankerson et al. describe a straight-forward way of adapting the ElGamal system to use with
Elliptic Curves (known as ECElGamal).\footnote{Hankerson et al. credit Koblitz and Miller with, independently, proposing
``using the group of points on an elliptic curve defined over a finite field to devise discrete
logarithm cryptographic schemes.''\cite{hankerson2010}} The sender \(A\) and recipient \(B\) agree on a curve \(E\)
(over a finite field \(\mathbb{F}_p\)) and a generator point \(G\). The prime order \(n\) of \(G\) must also be
known.

The recipient then computes a public key \(Q_B = d_B G\) (a point) from a randomly selected secret key
\(d_B \in_R [1,n-1]\) (an integer), and distributes the public key to the sender. \(A\) can now encrypt
a plaintext \(m\), encoded as a point \(M\) on the curve, in such a way that only \(B\) can decrypt
it. This is done by selecting a random \(k \in_R [1,n-1]\) and computing the ciphertext \(c\), a
tuple of two points (notice how the exponentiation of traditional ElGamal has been replaced with scalar
multiplication of points):

\begin{equation}
	c = (C_1, C_2)  \textbf{ where }  C_1 = kG \land C_2 = M+kQ_B
\end{equation}

The recipient can now decrypt the message by using his secret key, \(d_B\), as \(kQ_B = kd_BG\):

\begin{equation}
	M = M + k Q_B - k d_B G = C_2 - d_B C_1
\end{equation}

The original plaintext point can now safely be decoded to the original plaintext message.\cite{hankerson2010}

While the Bouncy Castle Java library contains an implementation of ECElGamal, the C\# Bouncy Castle library does not.