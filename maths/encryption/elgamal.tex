\subsubsection{ElGamal}

In 1985, ElGamal was proposed as a public key system, including a key distribution scheme based
on the difficulty of computing discrete logarithms. It allows messages to be sent from a sender
\(A\) to a recipient \(B\) to be encrypted in such a way that only \(B\) can decrypt them.

The sender and recipient agree on a common finite field (over a large prime \(p\)) to work in,
and use the same generator \(g\) for that field. \(B\) generates a secret key \(x_B\) and a public
key \(h_B = g^{x_B}\). The public key is distributed to the sender. Given a message \(m\), the
sender can now construct a ciphertext \(c\) from a randomly selected \(k\):

\begin{equation}
	c = (c_1, c_2)  \text{     where  }  c_1 = g^k \land c_2 = m * h_B^k
\end{equation}

\(B\) can find \(h_B^k\) by using \(c_1\) and his own secret key:

\begin{equation}
	h_B^k = g^{x_B*k} = c_1^{x_B}
\end{equation}

Finding \(m\) is then as simple as \(m = c_2/h_B^k\). \cite{elgamal} (p.1)

\paragraph{Adapting to Elliptic Curves}
\cite{hankerson2010} describes a straight-forward way of adapting the ElGamal system to use with
Elliptic Curves.\footnote{Hankerson et.al. credit Koblitz and Miller with, independently, proposing
"using the group of points on an elliptic curve defined over a finite field to devise discrete
logarithm cryptographic schemes." (p. 21)} The sender \(A\) and recipient \(B\) agree on a curve \(E\)
(over a finite field \(\mathbb{F}_p\)) and a generator point \(P\). The prime order \(n\) of \(P\) is
known, as well.

The recipient then computes a public key \(Q_B\) (a point) from a randomly selected secret key
\(d_B \in_R [1,n-1]\) (an integer), and distributes the public key to the sender. \(A\) can now encrypt
a plaintext \(m\), represented as a point \(M\) on the curve, in such a way that only \(B\) can decrypt
it. This is done by selecting a random \(k \in_R [1,n-1]\) and the computing the ciphertext \(c\), a
tuple of two points:

\begin{equation}
	c = (C_1, C_2)  \text{     where  }  C_1 = kP \land C_2 = M+kQ
\end{equation}

The recipient can now decrypt the message by using his secret key, \(d_B\):

\begin{equation}
	M = C_2 - d_B C_1
\end{equation}