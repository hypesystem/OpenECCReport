\subsection{Curves and Point Arithmetic}
\label{sec:math_curve}

As point multiplication is the action that takes the most time to perform,\cite{safecurves} several algorithms exist to
speed this up. Some of these will be discussed here (Section \ref{sec:math_curve_multiplication}). While only Weierstrass
curves are discussed in detail, the differences between Weierstrass curves and curves of other forms are touched upon briefly
(Section \ref{sec:math_curve_curves}).

Conceptually, the addition of two points, \(P + Q = R\), is done by drawing a line through the points, \(P\) and \(Q\),
and finding the point \(-R\), where the line intersects the curve again. This point is
then reflected over the x-axis to find \(R\), the result.

\begin{figure}[htb]
	\centering
	\includegraphics{maths/addition}
	\includegraphics{maths/doubling}
	\caption{Conceptual addition (left) and doubling (right) of points on elliptic curves.}
\end{figure}

There are a few cases that are not explained by this conceptual model: if a point is added to its reflection, \(P + (-P)\),
the line drawn through the two points will not intersect the curve at any other point. The result of such an addition is
the point at infinity, \(O_\infty\). An addition where one point is the point at infinity will be the point that is not
infinity: \(P + O_\infty = P\).

The double of \(P\), \(2P\) or \(P+P\), is found by drawing the tangent of the curve at \(P\).
\(-R\) is once again found at the line's (in this case the tangent's) next intersection
with the curve, and the result \(R\), is found by reflecting \(-R\).\cite{hankerson2010}

An elliptic curve \(E\) defined over a prime field \(\mathbb{F}_p\) can be reduced to the following form,
called the \emph{simple Weierstrass form}:

\begin{equation}
	E: y^2 = x^3 + ax + b
\end{equation}

The discriminant \(\Delta = 4a^3 + 27b^2 \neq 0\) ensures that each point in the curve has its own
unique tangent line. Such curves allow for simple point arithmetic: the
calculations are simpler, compared to other forms of curves used. The Weierstrass forms are
the only forms of curves supported by Bouncy Castle.\footnote{Bouncy Castle, at the time of writing,
supports \emph{FpCurve} over a prime field \(\mathbb{F}_p\) with any prime \(p\), and \emph{F2mCurve}
-- in both non-supersingular and supersingular forms -- over fields \(\mathbb{F}_{2 ^ m}\) with any \(m\).
The latter of these have the forms \(y^2 + xy = x^3 + ax^2 + b\) and
\(y^2 + cy = x^3 + ax + b\), and neither are supported in \emph{OpenECC} nor explained in any kind
of detail in this report. All three of these forms are Weierstrass simple forms.\cite{bouncycastle}}

On a curve of this form arithmetic consists of a few, very approachable calculations.
\paragraph{Addition}

Addition of two points \(P + Q = R\) is a fairly straight-forward matter. Given two points, \(P = (x_1,y_1)\) and
\(Q = (x_2,y_2)\), the result \(R = (x_3,y_3)\) can be found with the following calculations:

\begin{equation}
	\lambda = {{y_2-y_1} \over {x_2-x_1}}
\end{equation}

\begin{equation}
	x_3 = \lambda^2 - x_1 - x_2 \textbf{ and } y_3 = \lambda (x_1 - x_3) - y_1
\end{equation}

As elliptic curves also contain the \(\infty\) (infinity) point, there is a single exception to this
rule: if either \(P\) or \(Q\) is infinity, the result of adding them will be infinity, too.

\begin{equation}
    Q = \infty \lor P = \infty \implies P + Q = \infty\
\end{equation}

\paragraph{Doubling}

If the points being added are equal to one another (\(P + Q = R \text{, where } P = Q \text{ and } P = (x_1,y_1)\)),
the result can be calculated with a simplified formula:

\begin{equation}
	\lambda = {{3x_1^2 + a} \over {2y_1}} \text{, where } a \text { is the parameter from the curve formula }
\end{equation}

\begin{equation}
	x_3 = \lambda^2 - 2x_1 \textbf{ and } y_3 = \lambda (x_1 - x_3) - y_1
\end{equation}

The resulting point is then \(R = (x_3, y_3)\).

\paragraph{Subtraction}

A subtraction \(P - Q = R\) can be expressed through addition and negation, as \(P - Q = P + (-Q)\). Negation of a given point
\(P = (x,y)\) is calculated as \(-P = (x,-y)\) (a reflection over the x-axis).

As with addition, negation is complicated by the existence of the infinity point:

\begin{equation}
	-\infty = \infty \textbf{ and } P - P = \infty
\end{equation}

The negation of infinity is infinity (negative infinity does not exist), and a point subtracted from itself is infinity.\cite{hankerson2010}

\paragraph{Scalar Multiplication}

Scalar multiplication of a point \(P\) and an integer \(d\) is calculated through repeated addition of the point to an accumulator.
Several algorithms to speed up this process exist. Multiplication is the slowest operation performed in ECC, and its performance should
be considered carefully.\footnote{The DoubleAndAdd method limits the number of additions necessary by doubling the accumulating result
whenever possible, instead of naively adding \(P\); \textbf{Note: Explain FpNaf...}. Both of these methods are supported in OpenECC, and
the implementation can easily be extended with more multiplication algorithms (see Section \ref{sec:implementation_multipliers}).}
\subsubsection{Montgomery Curves}

Note: Explain in more detail the safety of curves.

Most safe curves\footnote{According to http://safecurves.cr.yp.to/} are not of the supported simple
Weierstrass form, but are rather represented as \emph{Edwards} or \emph{Montgomery} curves. Both of
these types of curves are more recent than the simple Weierstrass forms, and have shown to have some
desireable properties\footnote{Source? Expand on this.}

Note: Why have I chosen Montgomery and not Edwards curves?

Montgomery curves have the following form:

\begin{equation}
	y^2 = x^3 + ax^2 + x
\end{equation}

It is possible to represent all curves in Weierstrass simple forms, and as such it is also possible to
transform any of the safe Montgomery curves to this form.\cite{safecurves}
This does not, however, come without a cost, as the Weierstrass form of a Montgomery curve will have
very large integers as its parameters, compared to those in the original Montgomery form. For example,
the Montgomery curve, \verb|Curve25519| has the equation \(y^2 = x^3+486662x^2+x\), but
translated into the Weierstrass simple form, it will have an equation that does not fit on the width of
a normal A4 page:\footnote{See the full calculations and equation in Appendix \ref{app:montgomery_weierstrass}.
{\bf Note: Appendix with calculations, due to Lorena!}}

\begin{equation}
	y^2 =
	x^3 +
	19298681539552699237261830834781317975544997444273427339909597334573241639236x +
	55751746669818908907645289078257140818241103727901012315294400837956729358436
\end{equation}

Note: What is "acceptable change in parameters"? (The "rule" that is used to transform curves to Weierstrass form.) (Guide to Elliptic Curve Cryptography)

Note: In addition to smaller parameters, Montgomery curves have other inherent factors that result in quicker
computations. List them here.

Note: Operations on this curve are less trivial. It is also not supported by BouncyCastle.

Note: Arithmetic section here :)

\paragraph{Addition}

\paragraph{Doubling}

\paragraph{Subtraction}