\subsection{Encoding}
\label{sec:math_encoding}

In order to encrypt a message using elliptic curves (in the likes of the adapted ElGamal cryptosystem, see Section \ref{sec:math_enc_elgamal},
but not in hybrid cryptosystems,\footnote{Hybrid systems use an underlying symmetric system to make encoding obsolete.} see Section
\ref{sec:math_enc_ecies}), a way must be found to encode them as points on the curve.

Note: Trivial to map to the correct numberspace, but non-trivial to make sure that it is a point.

\subsubsection{Probablistic Mapping}

Koblitz is attributed with an algorithm that maps a message \(m\), represented as an integer in \(\mathbb{Z}_p\), to a point on the curve of the
simple Weierstrass form \(y^2 = x^3 + ax + b \text{ mod } p\).

An integer K is selected such that \((M + 1)K < p\), and this is used to compute
a candidate x-coordinate for the point, \(x_j = MK + j \text{ mod } p\). To verify the validity of the candidate, a check is performed: \(x_j\)
is a valid x-coordinate for a point on the curve if \(z_j = x_j^3 + ax_j + b\) has a square root. If the candidate is valid, the message \(m\)
can be mapped to a point \(M\) on the curve:

\begin{equation}
	M = (x_j, y_j) \text{, where } y_j = \sqrt{z_j} \text{ mod } p
\end{equation}

If no square root exists for \(z_j\) for all \(j\), \(j = 0 \text{ to } K-1\), then it is not possible to map the message to a point. In other
words, the method is probablistic and it is possible that a message cannot be encoded.\cite{MappingAMessage}

The point can be decode again by computing \(m = {\lfloor {{x} \over {K}} \rfloor}\).

Note: Largest possible K can be found as \(K = \lfloor {{p} \over {m + 1}} \rfloor\).

Note: Performance related to size of K should be discussed here. K is used in multiplication (inefficient for large K!), probability of finding
a mapping depends on K (more likely the larger K). Discuss tradeoff! (See MappingAPoint.)

Note: Messages can at most have a certain length. Any greater than p and they are not decodeable!

Note: Consider using the source \cite{MappingAMessage} uses instead of it (go to the sources), as the paper is actually about improving the mapping
for certain types of points. The original source is "W. Trappe and L. C. Washington, \emph{Introduction to Cryptography with Coding Theory}. 2nd Edition,
Prentice Hall, 2006."

\subsubsection{InjectiveEncodings}

--- injective encoding section is notes ---

\begin{quote}
    For certain groups \(\mathbb{G}\), like ... supersingular elliptic curves, it is not difficult to construct injective
	encodings...
\end{quote}

(from p.2) Supersingular curves are \(F_2m\) curves (not supported by us).

2.4: One can construct a probablistic injective encoding with size equal to about half of the size of \(\mathbb{G}\) (p.2)