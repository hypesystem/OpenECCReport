\section{Mathematical Foundation}

There are several distinct mathematical constructs required for building a working encryption scheme using Elliptic
Curve Cryptography. Elliptic curve arithmetic is a subset of finite field arithmetic, but as the latter is outside
of the scope of this project, an external source was used (see Section \ref{sec:implementation__dependencies}) and
the mathematics underlying finite fields will not be discussed here.

\textbf{Note: Add illustration of the arithmetic, and introduction here that explains that we are about to read
the theoretical workings of the arithmetic (before delving into how these are calculated in different
sources).}

Conceptually, the addition of two points, \(P + Q = R\), is done by drawing a line through the points, \(P\) and \(Q\),
and finding the point \(-R)\), where the line intersects the curve again. This point is
then reflected over the x-axis to find \(R\), the result.

The double of \(P\), \(2P\) or \(P+P\), is found by drawing the tangent of the curve at \(P\).
\(R_{reflection}\) is once again found at the line's (in this case the tangent's) next intersection
with the curve, and the result \(R\), is found by reflecting \(-R\).\cite{hankerson2010}

In the following, the mathematical constructs required for encryption are shown: artihmetic with points on the simplest
form of curves (the Weierstrass simple form, Section \ref{sec:math_curve}), probablistic encoding of a string message
to a point on a given curve (Section \ref{sec:math_encoding}), and encryption of a point on a curve (using a version
of the ElGamal public key encryption scheme that has been adapted to elliptic curves, Section \ref{sec:math_encryption}).

\subsection{Curves and Point Arithmetic}
\label{sec:math_curve}

An elliptic curve \(E\) defined over a prime field \(\mathbb{F}_p\) can be reduced to the following form,
called the \emph{simple Weierstrass form}:

\begin{equation}
	E: y^2 = x^3 + ax + b
\end{equation}

The discriminant \(\Delta = 4a^3 + 27b^2 \neq 0\) ensures that each point in the curve has its own
unique tangent line. Such curves allow for simple point arithmetic: the
calculations are simpler, compared to other forms of curves used. The Weierstrass forms are
the only forms of curves supported by Bouncy Castle.\footnote{Bouncy Castle, at the time of writing,
supports \emph{FpCurve} over a prime field \(\mathbb{F}_p\) with any prime \(p\), and \emph{F2mCurve}
-- in both non-supersingular and supersingular forms -- over fields \(\mathbb{F}_{2 ^ m}\) with any \(m\).
The latter of these have the forms \(y^2 + xy = x^3 + ax^2 + b\) and
\(y^2 + cy = x^3 + ax + b\), and neither are supported in \emph{OpenECC} nor explained in any kind
of detail in this report. All three of these forms are Weierstrass simple forms.\cite{bouncycastle}}

On a curve of this form arithmetic consists of a few, very approachable calculations.
\paragraph{Addition}

Addition of two points \(P + Q = R\) is a fairly straight-forward matter. Given two points, \(P = (x_1,y_1)\) and
\(Q = (x_2,y_2)\), the result \(R = (x_3,y_3)\) can be found with the following calculations:

\begin{equation}
	\lambda = {{y_2-y_1} \over {x_2-x_1}}
\end{equation}

\begin{equation}
	x_3 = \lambda^2 - x_1 - x_2 \textbf{ and } y_3 = \lambda (x_1 - x_3) - y_1
\end{equation}

As elliptic curves also contain the \(\infty\) (infinity) point, there is a single exception to this
rule: if either \(P\) or \(Q\) is infinity, the result of adding them will be infinity, too.

\begin{equation}
    Q = \infty \lor P = \infty \implies P + Q = \infty\
\end{equation}

\paragraph{Doubling}

If the points being added are equal to one another (\(P + Q = R \text{, where } P = Q \text{ and } P = (x_1,y_1)\)),
the result can be calculated with a simplified formula:

\begin{equation}
	\lambda = {{3x_1^2 + a} \over {2y_1}} \text{, where } a \text { is the parameter from the curve formula }
\end{equation}

\begin{equation}
	x_3 = \lambda^2 - 2x_1 \textbf{ and } y_3 = \lambda (x_1 - x_3) - y_1
\end{equation}

The resulting point is then \(R = (x_3, y_3)\).

\paragraph{Subtraction}

A subtraction \(P - Q = R\) can be expressed through addition and negation, as \(P - Q = P + (-Q)\). Negation of a given point
\(P = (x,y)\) is calculated as \(-P = (x,-y)\) (a reflection over the x-axis).

As with addition, negation is complicated by the existence of the infinity point:

\begin{equation}
	-\infty = \infty \textbf{ and } P - P = \infty
\end{equation}

The negation of infinity is infinity (negative infinity does not exist), and a point subtracted from itself is infinity.\cite{hankerson2010}

\paragraph{Scalar Multiplication}

Scalar multiplication of a point \(P\) and an integer \(d\) is calculated through repeated addition of the point to an accumulator.
Several algorithms to speed up this process exist. Multiplication is the slowest operation performed in ECC, and its performance should
be considered carefully.\footnote{The DoubleAndAdd method limits the number of additions necessary by doubling the accumulating result
whenever possible, instead of naively adding \(P\); \textbf{Note: Explain FpNaf...}. Both of these methods are supported in OpenECC, and
the implementation can easily be extended with more multiplication algorithms (see Section \ref{sec:implementation_multipliers}).}
\subsubsection{Montgomery Curves}

Note: Explain in more detail the safety of curves.

Most safe curves\footnote{According to http://safecurves.cr.yp.to/} are not of the supported simple
Weierstrass form, but are rather represented as \emph{Edwards} or \emph{Montgomery} curves. Both of
these types of curves are more recent than the simple Weierstrass forms, and have shown to have some
desireable properties\footnote{Source? Expand on this.}

Note: Why have I chosen Montgomery and not Edwards curves?

Montgomery curves have the following form:

\begin{equation}
	y^2 = x^3 + ax^2 + x
\end{equation}

It is possible to represent all curves in Weierstrass simple forms, and as such it is also possible to
transform any of the safe Montgomery curves to this form.\cite{safecurves}
This does not, however, come without a cost, as the Weierstrass form of a Montgomery curve will have
very large integers as its parameters, compared to those in the original Montgomery form. For example,
the Montgomery curve, \verb|Curve25519| has the equation \(y^2 = x^3+486662x^2+x\), but
translated into the Weierstrass simple form, it will have an equation that does not fit on the width of
a normal A4 page:\footnote{See the full calculations and equation in Appendix \ref{app:montgomery_weierstrass}.
{\bf Note: Appendix with calculations, due to Lorena!}}

\begin{equation}
	y^2 =
	x^3 +
	19298681539552699237261830834781317975544997444273427339909597334573241639236x +
	55751746669818908907645289078257140818241103727901012315294400837956729358436
\end{equation}

Note: What is "acceptable change in parameters"? (The "rule" that is used to transform curves to Weierstrass form.) (Guide to Elliptic Curve Cryptography)

Note: In addition to smaller parameters, Montgomery curves have other inherent factors that result in quicker
computations. List them here.

Note: Operations on this curve are less trivial. It is also not supported by BouncyCastle.

Note: Arithmetic section here :)

\paragraph{Addition}

\paragraph{Doubling}

\paragraph{Subtraction}
\subsection{Encoding}
\label{sec:math_encoding}

In order to encrypt a message using elliptic curves (in the likes of the adapted ElGamal cryptosystem, see Section \ref{sec:math_enc_elgamal},
but not in hybrid cryptosystems,\footnote{Hybrid systems use an underlying symmetric system to make encoding obsolete.} see Section
\ref{sec:math_enc_ecies}), a way must be found to encode them as points on the curve.

Note: Trivial to map to the correct numberspace, but non-trivial to make sure that it is a point.

\subsubsection{Probablistic Mapping}

Koblitz is attributed with an algorithm that maps a message \(m\), represented as an integer in \(\mathbb{Z}_p\), to a point on the curve of the
simple Weierstrass form \(y^2 = x^3 + ax + b \text{ mod } p\).

An integer K is selected such that \((M + 1)K < p\), and this is used to compute
a candidate x-coordinate for the point, \(x_j = MK + j \text{ mod } p\). To verify the validity of the candidate, a check is performed: \(x_j\)
is a valid x-coordinate for a point on the curve if \(z_j = x_j^3 + ax_j + b\) has a square root. If the candidate is valid, the message \(m\)
can be mapped to a point \(M\) on the curve:

\begin{equation}
	M = (x_j, y_j) \text{, where } y_j = \sqrt{z_j} \text{ mod } p
\end{equation}

If no square root exists for \(z_j\) for all \(j\), \(j = 0 \text{ to } K-1\), then it is not possible to map the message to a point. In other
words, the method is probablistic and it is possible that a message cannot be encoded.\cite{MappingAMessage}

Note: Consider using the source \cite{MappingAMessage} uses instead of it (go to the sources), as the paper is actually about improving the mapping
for certain types of points. The original source is "W. Trappe and L. C. Washington, \emph{Introduction to Cryptography with Coding Theory}. 2nd Edition,
Prentice Hall, 2006."

\subsubsection{InjectiveEncodings}

--- injective encoding section is notes ---

\begin{quote}
    For certain groups \(\mathbb{G}\), like ... supersingular elliptic curves, it is not difficult to construct injective
	encodings...
\end{quote}

(from p.2) Supersingular curves are \(F_2m\) curves (not supported by us).

2.4: One can construct a probablistic injective encoding with size equal to about half of the size of \(\mathbb{G}\) (p.2)
\subsection{Encryption}

(Rationale for choosing pub-key encryption?)

\subsubsection{ElGamal}

ElGamal is -- history -- public key cryptography -- stuff.

Traditional ElGamal uses pub and priv keys for... (maths!)

This can be adapted to elliptic curves, easily, by using ---

Differences are...

Required maths---
\paragraph{ECIES Encryption Implementation}

--- description of classes used ---

\begin{figure}[ht!]
	\centering
	\includegraphics[width=90mm]{img/naive_implementation__encryption__ecies_class_diagram.png}
	\caption{Class Diagram of structure around the ECIES encryption implementation}
	\label{ecies_class_diagram}
\end{figure}

This adds new dependencies to the project: AES and HMAC.

C\# has built-in support for Rijndael\footnote{http://msdn.microsoft.com/en-us/library/system.security.cryptography.rijndaelmanaged.aspx}
(AES)\footnote{See also: http://msdn.microsoft.com/en-us/magazine/cc164055.aspx}, and trusting this implementation is assumed safe for
the scope of this project. Again, the importance of the naive implementation is to show a full, working stack with naive implementations
of the basics of Elliptic Curves (but not other basic constructs).

It is possible to set a specific key in \verb+RijndaelManaged+, so what is now needed is the key derivation function, which will have to
be constructed for the purpose.

--- key deriv func ---

HMAC is supported in C\# as well\footnote{http://msdn.microsoft.com/en-us/library/system.security.cryptography.hmac.aspx; example: http://social.msdn.microsoft.com/Forums/vstudio/en-US/c963042a-24d3-45bb-ad00-272c00ed0bd4/encryptdecrypt-using-hmac-algorithm-in-c?forum=csharpgeneral}.

--- actual encryption ---